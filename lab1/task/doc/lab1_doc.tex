\documentclass[a4paper, 12pt]{scrartcl} 
\usepackage[english]{babel}
\usepackage[utf8]{inputenc}
\usepackage[T1]{fontenc}
\usepackage{ae,aecompl}

\usepackage{amsmath,amssymb,amstext}
\usepackage{psfrag}
\usepackage{float}
\usepackage[automark]{scrlayer-scrpage}

\usepackage{ifpdf}
\ifpdf
\usepackage[pdftex]{graphicx}
\usepackage{caption}
\usepackage{subcaption}
\pdfcompresslevel=9
\usepackage[pdftex=true, backref, pagebackref=false, colorlinks=true, bookmarks=true, bookmarksopen=false, bookmarksnumbered=false, pdfpagemode=None]{hyperref}
\DeclareGraphicsExtensions{.pdf}
\fi

\hypersetup{
pdftitle={}, 
pdfauthor={}, 
pdfsubject={}, 
pdfcreator={Accomplished with LaTeX2e and pdfLaTeX with hyperref-package.}, 
pdfproducer={}, 
pdfkeywords={} 
}
\newcommand{\mygraphics}[3]{
\begin{center}
\includegraphics[width=#1, keepaspectratio=true]{#2} \\
\textbf{#3}
\end{center}
}
\pagestyle{scrheadings}

\begin{document}
\section{Description Lab 1}

\subsection{Task 1}
\subsubsection{Subtask 1a}
Insert $x_w = v_w = \frac{dv_w}{dt} = i_A = 0$ into equation 3 of the mathematical model. \\
\begin{equation}
	\frac{dv_w}{dt} = 0 = \frac{1}{m_w + m_s(1-\cos(\phi)^2)} (m_s l \omega^2 \sin(\phi) - m_s g \sin(\phi) \cos(\phi))
\end{equation}
\begin{equation}
	m_s l \omega^2 \sin(\phi) = m_s g \sin(\phi) \cos(\phi)
\end{equation}
\begin{equation}
	l \omega^2 = g \cos(\phi)
\end{equation}
Than change the fourth equation of the mathematical model to get the form $\dot{\omega}(t) = p_1 f1(\phi(t), \omega(t), \dot{\omega}(t))$.
\begin{equation}
	\frac{d\omega}{dt} = \frac{1}{l\left(1-\frac{m_s}{m_w + m_s} \cos(\phi)^2\right)} \left( g \sin(\phi) - \frac{m_s}{m_w + m_s} \underbrace{l \omega^2}_{g \cos(\phi)} \sin(\phi) \cos(\phi) \right)
\end{equation}
\begin{equation}
	\frac{d\omega}{dt} = \frac{1}{l\left(1-\frac{m_s}{m_w + m_s} \cos(\phi)^2\right)} \left( g \sin(\phi) \left(1-\frac{m_s}{m_w + m_s} \cos(\phi)^2\right) \right)
\end{equation}
\begin{equation}
	\frac{d\omega}{dt} = \frac{g}{l} \sin(\phi)
\end{equation}
\subsubsection{Subtask 1b}
\subsubsection{Subtask 1c}
\subsubsection{Subtask 1d}

\subsection{Task 2}
\subsubsection{Subtask 2a}
This is the 4th equation of the mathematical model:
\begin{equation}
	\dot{\omega} = \frac{1}{l\left(1-\frac{m_s}{m_w + m_s} \cos(\phi)^2\right)} \left[ g \sin(\phi) -  \frac{\cos(\phi)F}{m_w + m_s} - \frac{m_s}{m_w + m_s} {l \omega^2} \sin(\phi) \cos(\phi) \right]
\end{equation}
By multiplying with the denominator we get: 
\begin{equation}
	\begin{split}
		\dot{\omega}l \left(1-\frac{m_s}{m_w + m_s} \cos(\phi)^2\right) = & g \sin(\phi) -  \frac{\cos(\phi)F}{m_w + m_s} - \frac{m_s}{m_w + m_s} {l \omega^2} \sin(\phi) \cos(\phi) \\
		\dot{\omega}l - g \sin(\phi) = & \frac{l m_s}{m_w + m_s} \dot{\omega} \cos(\phi)^2 - \frac{V}{m_w + m_s} \cos(\phi) i_A + \\
		& \frac{k_1}{m_w + m_s} \cos(\phi) v_w - \frac{l m_s}{m_w + m_s} \omega^2 \sin(\phi) \cos(\phi) \\
		\dot{\omega}l - g \sin(\phi) = & \frac{l m_s}{m_w + m_s} \left( \dot{\omega} \cos(\phi)^2 - \omega^2 \sin(\phi) \cos(\phi) \right) \\ 
		& - \frac{V}{m_w + m_s} \cos(\phi) i_A + \frac{k_1}{m_w + m_s} \cos(\phi) v_w \\
	\end{split}
\end{equation}
The 3 additive functions with its constant prefactors are: 
\begin{equation}
	p_1 = \frac{l m_s}{m_w + m_s} \text{ ; } f_1(\dot{\omega}, \phi, \omega) = \dot{\omega} \cos(\phi)^2 - \omega^2 \sin(\phi) \cos(\phi)
\end{equation}
\begin{equation}
	p_2 = \frac{V}{m_w + m_s}  \text{ ; } f_2(\phi, i_A) = -\cos(\phi) i_A
\end{equation}
\begin{equation}
	p_3 = \frac{k_1}{m_w + m_s} \text{ ; } f_3(\phi, v_w) = \cos(\phi) v_w
\end{equation}

\subsubsection{Subtask 2b}
\subsubsection{Subtask 2c}
\begin{equation}
	\begin{split}
		p_1 = & \frac{l m_s}{m_w + m_s} \\
		p_2 = & \frac{V}{m_w + m_s} \\
		p_3 = & \frac{k_1}{m_w + m_s} \\
	\end{split}
\end{equation}
$\rightarrow$ three equations but 4 unknowns, thus not possible. Therefore, we use $m_s=0.5kg$. 
\begin{equation}
	\begin{split}
		m_w = & \left(\frac{l}{p_1} - 1\right)m_s \\
		V = & (m_w + m_s)p_2 \\
		k_1 = & (m_w + m_s)p_3 \\
	\end{split}
\end{equation}
\subsubsection{Subtask 2d}

When using the small pendulum instead of the large one little to nothing changes for the parameter identification. 

The equation for $F$ in the system is simply replaced
\begin{equation}
	F = V i_A - k_1 v_w
\end{equation}
by the definition for the other force 
\begin{equation}
	F = \frac{k_M k_G}{r R_A} u - \frac{k_M^2 k_G^2}{r^2 R_A} v_w
\end{equation}
Both forces are calculated by multiplying the input ($u$ or $i_A$) and the velocity of the cart $v_W$ with a constant factor. In the second case, the factor consists of multiple physical constants that can than be abbreviated again by a constant. To estimate a specific physical constants, some other physical constants have to be known.

\subsubsection{Subtask 2e}
\subsubsection{Subtask 2f}

\subsection{Task 3}

\end{document}